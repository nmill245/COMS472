%%%%%%%%%%%%%%%%%%%%%%%%%%%%%%%%%%%%%%%%%%%%%%%%%%%%%%%%%%%%%%%%%%%%%%%%%%%%%%%%
%2345678901234567890123456789012345678901234567890123456789012345678901234567890
%        1         2         3         4         5         6         7         8

\documentclass[letterpaper, 10 pt, conference]{ieeeconf}   % Comment this line out if you need a4paper

%\documentclass[a4paper, 10pt, conference]{ieeeconf}      % Use this line for a4 paper

\IEEEoverridecommandlockouts                              % This command is only needed if 
                                                          % you want to use the \thanks command

\overrideIEEEmargins                                      % Needed to meet printer requirements.

%In case you encounter the following error:
%Error 1010 The PDF file may be corrupt (unable to open PDF file) OR
%Error 1000 An error occurred while parsing a contents stream. Unable to analyze the PDF file.
%This is a known problem with pdfLaTeX conversion filter. The file cannot be opened with acrobat reader
%Please use one of the alternatives below to circumvent this error by uncommenting one or the other
%\pdfobjcompresslevel=0
%\pdfminorversion=4

% See the \addtolength command later in the file to balance the column lengths
% on the last page of the document

% The following packages can be found on http:\\www.ctan.org
%\usepackage{graphics} % for pdf, bitmapped graphics files
%\usepackage{epsfig} % for postscript graphics files
%\usepackage{mathptmx} % assumes new font selection scheme installed
%\usepackage{times} % assumes new font selection scheme installed
%\usepackage{amsmath} % assumes amsmath package installed
%\usepackage{amssymb}  % assumes amsmath package installed

\title{\LARGE \bf
Project III Report for COM S 4/5720 Spring 2025: A Q-Learing Implemenation
}


\author{Noah Miller$^{1}$% <-this % stops a space
\thanks{$^{1}$Noah Miller with the Department of Computer Science, Iowa State University,
        Ames, IA 50013, USA
        {\tt\small nvmiller@iastate.edu}}%
}


\begin{document}



\maketitle
\thispagestyle{empty}
\pagestyle{empty}


%%%%%%%%%%%%%%%%%%%%%%%%%%%%%%%%%%%%%%%%%%%%%%%%%%%%%%%%%%%%%%%%%%%%%%%%%%%%%%%%
\begin{abstract}

	The problem of 3 chasing agents is difficult. These three agents, Tom, Jerry, and Spike, where Spike wants to attack Tom, Jerry wants to attack Spike, and Tom wants to attack Jerry.
	Not only is this problem being solved, but there is a chance that any action taken may be changed by an unkown probability.
	This paper details an implementation of Q-Learning as the planning algorithm of choice for these agents, as well as developments made along the way.

\end{abstract}


%%%%%%%%%%%%%%%%%%%%%%%%%%%%%%%%%%%%%%%%%%%%%%%%%%%%%%%%%%%%%%%%%%%%%%%%%%%%%%%%
%\section{COM S 4/5720 INSTRUCTIONS}
%Unless noted otherwise (mostly in this section), please follow the rules set by the instructions in the following sections.

%The submitted project report should be a compiled typeset pdf file using this template. The pdf is at most 3 pages (including references).

%Please give clear, rigorous, and correct descriptions about your proposed algorithm. Please also make sure your description aligns with your code. Please include appropriate references, if applicable (e.g., A-star algorithm~\cite{hart1968formal}).
%%%%%%%%%%%%%%%%%%%%%%%%%%%%%%%%
\section{INTRODUCTION}

It is common for a path seeking problem involving agents to be used when understanding machine learning. Once past the basic understanding of one agent attempting
to find the shortest path to a goal, is to introduce a moving goal. One step past that is to include someone to avoid as well. Once all that is met, we are left with the three agent problem. This
is a problem in which the three agents, Tom, Jerry, and Spike, attempt to capture each other whilst avoiding obstacles on the map. Each round the map is different, so it needs to allow variance.
From these turn, the things to consider are what moves are possible, and that Spike wants to capture Tom, Tom wants to capture Jerry, and Jerry wants to capture Spike. These considerations
change the problem from a simplest shortest path algorithm, to one needing more heuristics and understandings of the "best" path forward. The ultimate goal is to not be captured,
but one can easily change the preference of actions based off scoring the winning states. These actions though, may be changed in an unknown probability by the managing algorith.
There are 3 changes, keeping the action the same, rotating to the left by 90-degree, and rotating to the right by 90-degree. As such the model must not calculate
an algorithm for shortest path, but also one that will be the most likely to succeed given the randomness.

\section{Algorithmic Development}

\subsection{Q-Learning}

\subsubsection{Implementation}

The template is used to format your paper and style the text. All margins, column widths, line spaces, and text fonts are prescribed; please do not alter them. You may note peculiarities. For example, the head margin in this template measures proportionately more than is customary. This measurement and others are deliberate, using specifications that anticipate your paper as one part of the entire proceedings, and not as an independent document. Please do not revise any of the current designations

\subsubsection{Testing}

\section{Discussion}


\section{CONCLUSIONS}


\addtolength{\textheight}{-12cm}   % This command serves to balance the column lengths
% on the last page of the document manually. It shortens
% the textheight of the last page by a suitable amount.
% This command does not take effect until the next page
% so it should come on the page before the last. Make
% sure that you do not shorten the textheight too much.

%%%%%%%%%%%%%%%%%%%%%%%%%%%%%%%%%%%%%%%%%%%%%%%%%%%%%%%%%%%%%%%%%%%%%%%%%%%%%%%%



%%%%%%%%%%%%%%%%%%%%%%%%%%%%%%%%%%%%%%%%%%%%%%%%%%%%%%%%%%%%%%%%%%%%%%%%%%%%%%%%



%%%%%%%%%%%%%%%%%%%%%%%%%%%%%%%%%%%%%%%%%%%%%%%%%%%%%%%%%%%%%%%%%%%%%%%%%%%%%%%%


%%%%%%%%%%%%%%%%%%%%%%%%%%%%%%%%%%%%%%%%%%%%%%%%%%%%%%%%%%%%%%%%%%%%%%%%%%%%%%%%



\end{document}
